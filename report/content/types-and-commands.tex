%!TEX root = ../report.tex

\section{Types and Commands}\label{sec:types-com}

\begin{notation}
A tilde above a symbol indicates a list. For example, $\tilde{n}$ could be a list of natural numbers.
\end{notation}

Arithmetic expressions, boolean expressions and commands in IMP are given by the types \verb|aexp|, \verb|bexp| and \verb|com|, respectively. In addition, we have introduced three new concepts to IMP: exceptions, heap and functions.

Variables, exceptions, and functions are identified by instances of the types \verb|id|, \verb|exid| and \verb|funid|, respectively. Values are given as all natural numbers $v\in\mathbb{N}$.

A state $\sigma$ is a function space $\verb|id| \rightarrow \mathbb{N}$

A heap $\omega$ is a partial function space $\mathbb{N} \rightharpoonup \mathbb{N}$

An exception $\varepsilon$ is defined as the disjoint union $(\verb|exid|\times\tilde{\mathbb{N}})\uplus\bot$

A function is represented by the triple $\verb|func|: \verb|com| \; \times \; \tilde{id} \; \times \; \verb|aexp|$

Let $\rho$ be a partial function space, 
$\verb|funid| \rightharpoonup \verb|func|$, 
such that 
$\rho = \{ id_i \rightarrow f_i | id_i : \verb|funid| \wedge f_i:\verb|func| \}$.

Below follows the definitions of arithmetic and boolean expressions, as well as all commands, including the ones we have added for IMP+\footnote{The notation for $allocate$, $read$ and $write$ differs slightly from the syntax in the code.}.

\begin{equation}
\verb|aexp| \triangleq \mathbb{N} \;|\; \verb|id| \;|\; \verb|aexp| + \verb|aexp| \;|\;  \verb|aexp| - \verb|aexp| \;|\;  \verb|aexp| * \verb|aexp|
\end{equation}

\begin{equation}
\verb|bexp| \triangleq True \;|\; False \;|\; \verb|aexp| = \verb|aexp| \;|\; \verb|aexp| \leq \verb|aexp| \;|\; \neg\verb|bexp| \;|\; \verb|bexp| \wedge \verb|bexp|
\end{equation}

\begin{equation}
\begin{alignedat}{2}
\verb|com| \triangleq & \textbf{SKIP} \;|\; & (skip) \\&
	\verb|id| ::= \verb|aexp| \;|\; & (assignment) \\&
	\verb|com|\; ;;\; \verb|com| \;|\; & (sequence) \\&
	\textbf{IFB}\; \verb|bexp|\; \textbf{THEN}\; \verb|com|\; \textbf{ELSE}\; \verb|com|\; \textbf{FI} \;|\; & (if-then-else) \\&
	\textbf{WHILE}\; \verb|bexp|\; \textbf{DO}\; \verb|com|\; \textbf{END}\; \;|\; & (while-loop) \\&
	\textbf{THROW}\; \verb|exid|, \tilde{aexp} \;|\; & (throw) \\&
	\textbf{TRY}\; \verb|com|\; \textbf{CATCH}\; \verb|exid|, \tilde{id}\; \textbf{DO}\; \verb|com|\; \textbf{END} \;|\; & (try-catch) \\&
	\textbf{CALL}\; \verb|funid|\; \verb|id|\; \tilde{aexp} \;|\; & (call) \\&
	\verb|id|\leftarrow:\;\textbf{ALLOC}\;|\; & (allocate) \\&
	\verb|aexp| \rightarrow \verb|id|\;|\; & (read) \\&
	\verb|aexp|\leftarrow\;\verb|aexp| \;|\; & (write) \\&
	\textbf{FREE}\;\verb|id| & (free)
\end{alignedat}
\end{equation}

{\scriptsize
Note, to increase clarity, some of the notation for allocation, read and write differs from the syntax in the Coq code. The syntax used in Coq is \verb|id <-# ALLOC|, \verb|id <-* [id]|, and \verb|[aexp] <-@ aexp|
}

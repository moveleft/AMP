%!TEX root = ../report.tex

\section{Hoare Rules}

Every Hoare rule in this section has been proven formally in Coq (The proofs can be found \verb|./src/Hoare.v|), with the exception of rules \ref{eqn:hoare-alloc}, \ref{eqn:hoare-read}, \ref{eqn:hoare-write} and \ref{eqn:hoare-free} which have been proven for a version of IMP that has not been extended with exceptions and function calls. The proofs for those four rules can be found in \verb|./sep-src/Hoare.v|.

\begin{defn}

Let $e$ be address af an allocated cell on the heap and $e'$ the value stored in the cell.

\begin{equation}
e \mapsto e'
\end{equation}

To state that an address points to an allocated cell without caring about the value, we write
\begin{equation}
e \mapsto \textbf{-}
\end{equation}

\end{defn}

\begin{defn}
\verb|emp| is defined to be an assertion which holds for the empty heap.
\end{defn}

\begin{defn}
\verb|empty| is defined to be the empty state. However, a state is a (complete) function space from natural numbers. When referring to an empty state, we actually refer to a state, $\sigma$, such that $\forall n \in \mathbb{N}.\; \sigma(n) = 0$.
\end{defn}

\begin{equation}\label{eqn:hoare-cons-pre}
\frac{\vdash\{P'\} \; c \; \{Q\} \qquad P \Rightarrow P'}{\vdash\{P\} \; c \; \{Q\}}
\end{equation}

\begin{equation}\label{eqn:hoare-cons-post}
\frac{\vdash\{P\} \; c \; \{Q'\} \qquad Q \Rightarrow Q'}{\vdash\{P\} \; c \; \{Q\}}
\end{equation}

\begin{equation}\label{eqn:hoare-assign}
\frac{}{\vdash \{Q[\sfrac{X}{a}]\} \; (X ::= a) \; \{Q \wedge \lambda \varepsilon\:\sigma\:\omega. \varepsilon = \bot\}}
\end{equation}

\begin{equation}\label{eqn:hoare-seq}
\frac{\vdash \{P\} \; c_1 \; \{Q \wedge \lambda \varepsilon\:\sigma\:\omega. \varepsilon = \bot\} \qquad \vdash \{Q\} \; c_2 \; \{R\}}{\vdash \{P\} \; (c_1\; ;;\; c_2) \; \{R\}}
\end{equation}

\begin{equation}\label{eqn:hoare-seq-exn}
\frac{\vdash \{P\} \; c_1 \; \{Q \wedge \lambda \varepsilon\:\sigma\:\omega. \varepsilon \neq \bot\}}{\vdash \{P\} \; (c_1\; ;;\; c_2) \; \{R\}}
\end{equation}

\begin{equation}
\frac{\vdash \{P \wedge b=true\} \; c_1 \; \{Q\} \qquad \vdash \{P \wedge b=false\} \; c_2 \; \{Q\}}{\vdash \{P\} \; (\textbf{IFB}\; b\; \textbf{THEN}\; c_1\; \textbf{ELSE}\; c_2\; \textbf{FI}) \; \{Q\}}
\end{equation}

\begin{equation}
\frac{\vdash \{P \wedge b = true\} \; c \; \{P\}}{\vdash \{P\} \; (\textbf{WHILE}\; b\; \textbf{DO}\; c\; \textbf{END}) \; \{P \wedge \lambda \varepsilon\:\sigma\:\omega.((\varepsilon = \bot) \Rightarrow (b = false))\}}
\end{equation}

\begin{equation}\label{eqn:hoare-throw}
\frac{\tilde{a}, \sigma \Downarrow \tilde{n}}{\vdash \{P\} \; (\textbf{THROW}\; e, \tilde{a}) \; \{P \wedge \lambda \varepsilon\:\sigma\:\omega. \varepsilon = (e, \tilde{n})\}}
\end{equation}

\begin{defn}
Let \verb|disjoint| be a function $(\varepsilon\times\texttt{exid})\rightarrow(\top\cup\bot)$, given by the following equation:
\begin{equation}
\texttt{disjoint}(\varepsilon, e) = \begin{cases}
\top & \mbox{if} \; \varepsilon = \bot \\
e \neq e' & \mbox{if} \; \varepsilon = (e', \_)
\end{cases}
\end{equation}
\end{defn}

\begin{equation}\label{eqn:hoare-try}
\frac{\vdash \{P\} \; c_1 \; \{Q \wedge \lambda \varepsilon\:\sigma\:\omega. \texttt{disjoint}(\varepsilon, e)\}}{\vdash \{P\} \; (\textbf{TRY}\; c_1\; \textbf{CATCH}\; e, \tilde{X}\; \textbf{DO}\; c_2\; \textbf{END}) \; \{Q\}}
\end{equation}

\begin{equation}\label{eqn:hoare-catch}
\frac{\vdash \{P\} \; c_1 \; \{\lambda \varepsilon\:\sigma\:\omega. \varepsilon = (e, \tilde{n})\} \qquad \vdash \{\lambda \varepsilon\:\sigma\:\omega. \exists \sigma'. (P(\varepsilon, \sigma', \omega) \wedge \sigma = \sigma'[\sfrac{\tilde{X}}{\tilde{n}}])\} \; c_2 \; \{Q\}}{\vdash \{P\} \; (\textbf{TRY}\; c_1\; \textbf{CATCH}\; e, \tilde{X}\; \textbf{DO}\; c_2\; \textbf{END}) \; \{Q\}}
\end{equation}

\begin{equation}\label{eqn:hoare-call}
F = (c,\tilde{p},r) \wedge \left\{P\right\} c \left\{Q \wedge w = r\right\} \vdash \left\{\tilde{v} = \tilde{a} \wedge P(\texttt{empty}[\sfrac{\tilde{v}}{\tilde{p}}])\right\}\textbf{CALL} \; F \; X \; \tilde{a} \left\{X = w\right\}
\end{equation}

\begin{equation}\label{eqn:hoare-alloc}
\frac{}{\vdash \{ \texttt{emp} \}\; X\leftarrow:\;\textbf{ALLOC}\; \{X \mapsto 0\}}
\end{equation}

\begin{equation}\label{eqn:hoare-read}
\frac{}{\vdash \{\; \lambda\varepsilon\;\sigma\;\omega.(a,\sigma)\Downarrow m \wedge \omega(m) = v\}\; a \rightarrow X \;\{ X = v \}}
\end{equation}

\begin{equation}\label{eqn:hoare-write}
\frac{}{\vdash \{ a_1 \mapsto X \}\; a_1 \leftarrow a_2 \;\{ a_1 \mapsto a_2 \}}
\end{equation}

\begin{equation}\label{eqn:hoare-free}
\frac{}{\vdash \{X \mapsto \textbf{-}\}\textbf{FREE}\;X \{\texttt{emp} \}}
\end{equation}

\section{Discussion}

\subsection{Design Choices}
The operational semantics for SEQUENCE carries the entire burden of propagating exceptions from one command to the next, in case the post-state of one, and thus the implicit pre-state of the next is exceptional. There are a few reasons for doing it this way:

\begin{itemize}
\item It simplifies the evaluation relation, since there is no need to explicitly model the concept of an exceptional pre-state
\item We avoid having to add an additional rule for every command that makes it explicitly behave as SKIP in the presence of an exception
\end{itemize}

This way, any command that can be put in a sequence will implicitly be skipped if the post-state of the first command is exceptional.

\subsection{Verified Examples}

\subsubsection{Caught and Propagated Exception}

\begin{lstlisting}[mathescape=true,keepspaces=true,label=lst:hoare_ex_asgn,caption=A small program demonstrating a try-catch statement.]
{ 3 = 3                         }      $\text{This is trivially true, but necessary due to}$
                                       $\text{how the assignment Hoare rule (\ref{eqn:hoare-assign}) works.}$
X ::= 3;;                              $\text{Updates the state with X = 3 per the}$
                                       $\text{operational semantics for assignment (\ref{eqn:op-assign}).}$
{ X = 3 $\wedge\quad\lambda \varepsilon\:\sigma\:\omega.\:\varepsilon=\bot$            }      $\text{Trivially true; the state has just been updated}$
                                       $\text{such that X = 3.}$
TRY
  { X = 3 $\wedge\quad\lambda \varepsilon\:\sigma\:\omega.\:\varepsilon=\bot$          } $\Rightarrow$   $\text{Precondition as per the catch Hoare rule (\ref{eqn:hoare-catch}).}$
  { 4 = 4 $\wedge\quad\lambda \varepsilon\:\sigma\:\omega.\:\varepsilon=\bot$          }      $\text{Implies this by rule of consequence, which is}$
                                       $\text{trivially true.}$
X ::= 4;;                              $\text{Updates the state with X = 4 per the}$
                                       $\text{operational semantics for assignment (\ref{eqn:op-assign}).}$
  { X = 4 $\wedge\quad\lambda \varepsilon\:\sigma\:\omega.\:\varepsilon=\bot$          }
  THROW T, [21 - X];;                  $\text{Puts the program in an exceptional state as}$
                                       $\text{per the operational semantics for throw (\ref{eqn:op-throw}).}$
  { $\lambda \varepsilon\:\sigma\:\omega.\:\varepsilon$ = (T, [17])          }      $\text{Follows from the Hoare rule for throw (\ref{eqn:hoare-throw}).}$
  THROW U, []                          $\text{This statement is skipped by way of the}$
                                       $\text{exception sequence Hoare rule (\ref{eqn:hoare-seq-exn}).}$
  { $\lambda \varepsilon\:\sigma\:\omega.\:\varepsilon$ = (T, [17])          }      $\text{I.e. the exception from before is propagated.}$
CATCH T, [Y] DO
  { $\lambda \varepsilon\:\sigma\:\omega.\:\exists \sigma'.(\sigma(X)=3\wedge\sigma=\sigma'[\sfrac{Y}{17}])$ } $\Rightarrow$    $\text{Follows from the catch Hoare rule (\ref{eqn:hoare-catch}).}$
                                       $\text{Any state in which it holds that X = 3}$
                                       $\text{will do as a witness for the existential}$
                                       $\text{quantifier.}$
  { X = 3 $\wedge$ Y = 17              } $\Rightarrow$    $\text{The rule of consequence leads us to this.}$
  { X + Y = 20 $\wedge$ Y = 17         }       $\text{Per the assignment Hoare rule (\ref{eqn:hoare-assign}).}$
  X ::= X + Y                           $\text{Updates the state with X = 20}$
  { X = 20 $\wedge$ Y = 17 $\wedge\quad\lambda \varepsilon\:\sigma\:\omega.\:\varepsilon=\bot$}
END
{ X = 20 $\wedge$ Y = 17 $\wedge\quad\lambda \varepsilon\:\sigma\:\omega.\:\varepsilon=\bot$  }      $\text{Follows from the catch Hoare rule (\ref{eqn:hoare-catch}).}$
                                        $\text{This is our desired end state.}$
\end{lstlisting}
\section{Basics of IMP}
IMP is a Turing-complete imperative language. The version of it that we are concerned with is the one implemented in Coq as part of the exercise chapters in Software Foundations. It consists of two types of expressions, arithmetic and boolean, in addition to commands (i.e. statements). Arithmetic and boolean expressions are evaluated recursively, and commands are evaluated with an inductive relation.

Formal definitions of arithmetic and boolean expressions can be found in appendix X. Definitions of the various commands as well as the operational semantics for the entirety of IMP+ follow in a later chapter, which will shed more light on the details of how each individual command is evaluated.
%!TEX root = ../report.tex

\section{Operational Semantics}\label{sec:opsem}

\begin{notation}
Let $c$ be a command and $\sigma'$, $\varepsilon$ and $\omega'$ be the state, exception and heap after the execution of $c$ in $\rho$, $\sigma$ and $\omega$.

This is written:
\begin{equation}
(c,\rho,\sigma,\omega) \Downarrow (\sigma',\epsilon,\omega')
\end{equation}

Of note is the fact that the environment $\rho$ remains constant before and after execution, so it is only present on the left-hand side of the evaluation.

Additionally, a program cannot begin execution in an exceptional state, but it can end in one. Thus, the exception $\varepsilon$ is only present on the right-hand side of the evaluation. Conceptually speaking, an individual command \emph{can} begin its execution in an exceptional state (and thus be skipped), but semantically this is modelled by the operational semantics of the $sequence$ command, and not directly in the evaluation relation.
\end{notation}

\begin{notation}
Let $a$ be an arithmetic expression and $n\in\mathbb{N}$ be the result of evaluating $a$ in $\sigma$.

This is written:
\begin{equation}
(a,\sigma) \Downarrow n
\end{equation}
\end{notation}

\begin{notation}
Let $b$ be a boolean expression and $\beta$ the boolean result of evaluating $b$ in $\sigma$.

This is written:
\begin{equation}
(b,\sigma) \Downarrow \beta
\end{equation}
\end{notation}

\begin{defn}\label{def:state-update}
Let $\sigma'$ be the state which is the result of updating the state $\sigma$ with $X = n$, where $X$ is an identifier and $n$ a natural number. It holds that

\begin{equation}
\begin{split}
&\sigma' = \sigma[\sfrac{X}{n}] =>\\
&\sigma'(X) = n \wedge (\forall Y.\; Y \neq X => \sigma'(Y) = \sigma(Y))
\end{split}
\end{equation}
\end{defn}

\begin{defn}\label{def:state-list-update}
Let $\sigma'$ be the state which is the result of updating the state $\sigma$ with $X_i = n_i$ for all $(X : \verb|id|) \in \tilde{X}$ and $n \in (\tilde{n} \cup \mathbb{N})$. It holds that

\begin{equation}
\begin{split}
&\sigma' = \sigma[\sfrac{\tilde{X}}{\tilde{n}}] =>\\
&(\forall X_i \in \tilde{X}.\; \sigma'(X_i) = n_i) \wedge (\forall Y.\; Y \notin \tilde{X} => \sigma'(Y) = \sigma(Y))
\end{split}
\end{equation}
\end{defn}

Below follows the operational semantics for all commands in IMP+, with explanatory text where it is relevant.

The skip command has the same semantics as in IMP. It has no effect on the state nor the heap and can not throw an exception.
\begin{equation}
\frac{}{(\textbf{SKIP}, \rho, \sigma, \omega) \Downarrow (\sigma, \bot, \omega)}
\end{equation}

The assignment has the same semantics as in IMP. It updates the state such that the specified variable contains the specified value.
\begin{equation}\label{eqn:op-assign}
\frac{(a, \sigma) \Downarrow n \qquad \sigma' = \sigma[\sfrac{X}{n}]}{((X ::= a), \rho, \sigma, \omega) \Downarrow (\sigma', \bot, \omega)}
\end{equation}

The if command has the same semantics as in IMP, though it can throw an exception iff the executed branch throws on. It is one of the two branching commands available in IMP+. The then-branch is executed iff the guard evaluates to true. Equivelently, the else-branch is executed iff the guard evaluates to false. The if command terminates in the same state, exception and heap as the executed branch. This is covered be operational semantic \ref{eqn:op-if-true} and \ref{eqn:op-if-false}.
\begin{equation}\label{eqn:op-if-true}
\frac{(b, \sigma) \Downarrow true \qquad (c_1, \rho, \sigma, \omega) \Downarrow (\sigma', \varepsilon, \omega')}{((\textbf{IFB}\; b\; \textbf{THEN}\; c_1\; \textbf{ELSE}\; c_2\; \textbf{FI}), \rho, \sigma, \omega) \Downarrow (\sigma', \varepsilon, \omega')}
\end{equation}

\begin{equation}\label{eqn:op-if-false}
\frac{(b, \sigma) \Downarrow false \qquad (c_2, \rho, \sigma, \omega) \Downarrow (\sigma', \varepsilon, \omega')}{((\textbf{IFB}\; b\; \textbf{THEN}\; c_1\; \textbf{ELSE}\; c_2\; \textbf{FI}), \rho, \sigma, \omega) \Downarrow (\sigma', \varepsilon, \omega')}
\end{equation}

The operational semantics for $sequence$ are largely the same as in IMP, although one additional case has been added. The original case (\ref{eqn:op-seq}) is the same, except for ending in an arbitrary exception, as with the if-statements. The second case (\ref{eqn:op-seq-exn}), however, is central to the handling of exceptions in IMP+. It is through this that commands which "begin" in an exceptional state, by virtue of the first leg of the sequence ending in one, are skipped.

\begin{equation}\label{eqn:op-seq}
\frac{(c_1, \rho, \sigma, \omega) \Downarrow (\sigma', \bot, \omega') \qquad (c_2, \rho, \sigma', \omega') \Downarrow (\sigma'', \varepsilon, \omega'')}{((c_1\; ;;\; c_2), \rho, \sigma, \omega) \Downarrow (\sigma'', \varepsilon, \omega'')}
\end{equation}

\begin{equation}\label{eqn:op-seq-exn}
\frac{(c_1, \rho, \sigma, \omega) \Downarrow (\sigma', \varepsilon, \omega') \qquad \varepsilon \neq \bot}{((c_1\; ;;\; c_2), \rho, \sigma, \omega) \Downarrow (\sigma', \varepsilon, \omega')}
\end{equation}

The operational semantics for the while-loop have been extended with a case (\ref{eqn:op-while-exn}) for breaking out of the loop if the inner code block results in an exception. The semantics simply state that if the condition is true, and the inner code block results in a state and some exception, the entire statement results in that state and exception. This is the same behavior as for the $break$ command that was implemented in IMP during one exercise session.

\begin{equation}
\frac{(b, \sigma) \Downarrow false}{((\textbf{WHILE}\; b\; \textbf{DO}\; c\; \textbf{END}), \rho, \sigma, \omega) \Downarrow (\sigma, \bot, \omega)}
\end{equation}

\begin{equation}
\frac{(b, \sigma) \Downarrow true \qquad (c, \rho, \sigma, \omega) \Downarrow (\sigma', \bot, \omega') \qquad ((\textbf{WHILE}\; b\; \textbf{DO}\; c\; \textbf{END}), \rho, \sigma', \omega') \Downarrow (\sigma'', \bot, \omega'')}{((\textbf{WHILE}\; b\; \textbf{DO}\; c\; \textbf{END}), \rho, \sigma, \omega) \Downarrow (\sigma'', \bot, \omega'')}
\end{equation}

\begin{equation}\label{eqn:op-while-exn}
\frac{(b, \sigma) \Downarrow true \qquad (c, \rho, \sigma, \omega) \Downarrow (\sigma', \varepsilon, \omega') \qquad \varepsilon \neq \bot}{((\textbf{WHILE}\; b\; \textbf{DO}\; c\; \textbf{END}), \rho, \sigma, \omega) \Downarrow (\sigma', \varepsilon, \omega')}
\end{equation}

\begin{equation}\label{eqn:op-throw}
\frac{(\tilde{a}, \sigma) \Downarrow \tilde{n}}{((\textbf{THROW}\; e,\; \tilde{a}), \rho, \sigma, \omega) \Downarrow (\sigma, (e, \tilde{n}), \omega)}
\end{equation}

The last case (\ref{eqn:op-catch}) for $try-catch$ is the interesting one. The catch clause contains the name of an exception and a list of variable names. If an exception with a matching name is thrown within the first code block, the values associated with the exception are assigned (in linear order) to the variable names in the list and the second block is executed from this updated version of the original state. Otherwise, the second code block is skipped (case \ref{eqn:op-try} and \ref{eqn:op-try-exn}).

\begin{equation}\label{eqn:op-try}
\frac{(c_1, \rho, \sigma, \omega) \Downarrow (\sigma', \bot, \omega')}{((\textbf{TRY}\; c_1\; \textbf{CATCH}\; e,\; \tilde{X}\; \textbf{DO}\; c_2\; \textbf{END}), \rho, \sigma, \omega) \Downarrow (\sigma', \bot, \omega')}
\end{equation}

\begin{equation}\label{eqn:op-try-exn}
\frac{e \neq e' \qquad (c_1, \rho, \sigma, \omega) \Downarrow (\sigma', (e, \tilde{n}), \omega')}{((\textbf{TRY}\; c_1\; \textbf{CATCH}\; e',\; \tilde{X}\; \textbf{DO}\; c_2\; \textbf{END}), \rho, \sigma, \omega) \Downarrow (\sigma', (e, \tilde{n}), \omega')}
\end{equation}

\begin{equation}\label{eqn:op-catch}
\frac{(c_1, \rho, \sigma, \omega) \Downarrow (\sigma', (e, \tilde{n}), \omega') \qquad (c_2, \rho, \sigma[\sfrac{\tilde{X}}{\tilde{n}}], \omega) \Downarrow (\sigma'', \varepsilon, \omega'')}{((\textbf{TRY}\; c_1\; \textbf{CATCH}\; e,\; \tilde{X}\; \textbf{DO}\; c_2\; \textbf{END}), \rho, \sigma, \omega) \Downarrow (\sigma'', \varepsilon, \omega'')}
\end{equation}

The call command consists of two identifiers, one for the function to call and one for the variable the return value is stored in, and a list of arithmetic expressions which are the arguments to the function.

As previously stated, a function is defined by a triple containing a code block, the body, a list of identifiers for the parameters, and an arithmetic expression (the return expression).

When a function is called, the arguments of the call are evaluated and a new state is constructed by assigning the argument values to the parameters of the function (in linear order), in the empty state. The body is executed in the constructed state. The return expression is evaluated in the state the execution of the body terminates in. The return value are stored in the variable for the same scope(state) of the call.

Note, the return expression is evaluated and stored in the specified variable regardless of whether an exception was the reason for the termination of the body or not. However, if the body terminates with an exception, then the call also terminates in the same exception.

The heap is the same in both the scope of the call and the body of the called function. Any effect on the heap in the body will persist after the call terminates, even if it terminates with an exception. But, in case of an exception, the catch clause will still discard any changes made to the heap in the corresponding try block.

\begin{equation}\label{eqn:op-call}
\frac
{(\tilde{a},\sigma) \Downarrow \tilde{v} \quad \sigma''= emp[\sfrac{\tilde{v}}{\tilde{p}}] \quad (c, \rho, \sigma'', \omega) \Downarrow (\sigma''', \varepsilon, \omega') \quad (r,\sigma''') \Downarrow w \quad \sigma' = \sigma[\sfrac{X}{w}] \quad \rho(f) = (c, \tilde{p}, r)}
{((\textbf{CALL}\; F \; X  \; \tilde{a}), \rho, \sigma, \omega) \Downarrow (\sigma', \varepsilon, \omega')}
\end{equation}

The allocation command is define only for a heap which have space for a new cell. It allocates a new call and initialises the cell's value to zero. The address of the allocated cell are stored in the specified variable.
\begin{equation}
\frac
{\omega(m)=\bot \qquad \omega'=\omega\cup\{m \mapsto 0\} \qquad \sigma' = \sigma[\sfrac{X}{m}]}
{((X\leftarrow:\;\textbf{ALLOC}\;), \rho, \sigma, \omega) \Downarrow (\sigma', \bot, \omega')}
\end{equation}

The read command is defined only for a heap which have a cell allocated with the address, that the specified aritmetic expression evaluates to. It looks up the cell at the address and stores the cell's value in the specified variable.
\begin{equation}
\frac
{(a,\sigma) \Downarrow m \qquad \omega(m)=v \qquad \sigma'=\sigma[\sfrac{X}{v}]}
{((a \rightarrow X), \rho, \sigma, \omega) \Downarrow (\sigma', \bot, \omega)}
\end{equation}

The write command is defined only for a heap which have a cell allocated with the address, that the first specified aritmetic expression evaluates to. The value of the cell located at the address as updated with the value the secind arithmetic expression evaluates to.
\begin{equation}
\frac
{(a,\sigma)\Downarrow m \qquad (a',\sigma)\Downarrow v \qquad \omega'=(\omega\setminus\{m\mapsto\omega(m)\})\cup\{m\mapsto v\}}
{((a\leftarrow\;a'), \rho, \sigma, \omega) \Downarrow (\sigma, \bot, \omega')}
\end{equation}

The free command is defined only for a heap which have a cell allocated with the address stored in the specified variable. The cell are deallocated and no read, write or free command can access the cell untill the allocation command has (re)allocated the particular cell.
\begin{equation}
\frac
{\sigma(X)=m \qquad \omega(m)\neq\bot \qquad \omega'=\omega\setminus\{m\mapsto\omega(m)\}}
{((\textbf{FREE}\;X), \rho, \sigma, \omega) \Downarrow (\sigma, \bot, \omega')}
\end{equation}

\begin{prop}\label{prop:program-correct}
Given all operational sematics shown in this chapter, we postulate that the IMP+ program from listing \ref{lst:imp-plus-ex} ends in a state where it holds that $X = 22$.
\end{prop}

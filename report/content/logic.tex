%!TEX root = ../report.tex

\subsection{Hoare Logic}
To reason about a code block we introduce Hoare logic. To this end we define an \verb|assertion| to be a predicate, $P$, which takes a state, an exception, and a heap.

Hoare triples are the core of Hoare logic. They allow us to describe the effects of executing a piece of code. A Hoare triple has the form $\{P\}\;c\;\{Q\}$ where $P,Q : \verb|assertion|$ and $c : \verb|com|$.
Equation \ref{eqn:hoare_triple_def} defines the Hoare triple.

\begin{equation}\label{eqn:hoare_triple_def}
\begin{alignedat}{1}
\{P\}\;c\;\{Q\} \triangleq &\forall \sigma \: \sigma' \: \omega \: \omega' \: \epsilon .\\
& (c,\sigma,\omega) \Downarrow (\sigma', \epsilon, \omega') \Rightarrow\\
& P(\sigma,\bot,\omega) \Rightarrow\\
& Q(\sigma',\epsilon,\omega')
\end{alignedat}
\end{equation}

Listing \ref{lst:hoare_ex_asgn} shows an example of using Hoare logic to verify a simple program in IMP.

\begin{lstlisting}[mathescape=true,keepspaces=true,label=lst:hoare_ex_asgn,caption=A simple code block proven using Hoare triples.]
{ 1 = 1               } $\text{This is trivially true, but necessary due to }$
                        $\text{how the assignment Hoare rule (\ref{eqn:hoare-assign}) work}$
X ::= 1                 $\text{Updates the state with X = 1 per the}$
                        $\text{operational semantics for assignment (\ref{eqn:op-assign})}$
{ X = 1               } $\text{Trivially true; the state has just been updated}$
                        $\text{such that X = 1}$
{ X = 1 $\wedge$ X $\cdot$ 10 = 10 } $\text{The right operand of the conjunction is}$
                        $\text{introduced to conform to the assignment Hoare rule}$
Y ::= X $\cdot$ 10
{ X = 1 $\wedge$      Y = 10 } $\text{True; the state has jst been updated such}$
                        $\text{that Y = 10}$
\end{lstlisting}

\subsection{Specification Logic}
The specification logic allows us to reason about functions and function calls. We introduce this logic by extending the Hoare logic.

The \verb|assertion| is extended such that it takes $\rho$ as an additional parameter. The Hoare triple is extended such that it is a function from $\rho$ to the Hoare triple defined in equation \ref{eqn:hoare_triple_def}.
\begin{equation}
\begin{alignedat}{2}
\{P\}\;c\;\{Q\} \triangleq \forall \sigma \: \sigma' \: \omega \: \omega' \: \epsilon .\\
 \lambda\rho.(
  && (c,\rho,\sigma,\omega) \Downarrow (\sigma', \epsilon, \omega') \Rightarrow\\
&& P(\rho,\sigma,\bot,\omega) \Rightarrow\\
&& Q(\rho,\sigma',\epsilon,\omega'))
\end{alignedat}
\end{equation}

Listing \ref{lst:spec-ex-body} through \ref{lst:spec-ex-prog} is an example of a program which calls a function and assigns the return value to a variable in the local state.

\begin{lstlisting}[mathescape=true,keepspaces=true,label=lst:spec-ex-body,caption=The body of the function F]
{ X + Y = X + Y }
Z := X + Y
{ Z = X + Y     }
\end{lstlisting}

\begin{lstlisting}[mathescape=true,keepspaces=true,label=lst:spec-ex-env,caption=A partial function space containing the function F.]
$\rho$ = $\{$F $\rightarrow$ (Z := X + Y, [X, Y], Z)$\}$
\end{lstlisting}

\begin{lstlisting}[mathescape=true,keepspaces=true,label=lst:spec-ex-prog,caption=A program which calls the function F and stores the result in X]
{ 3 = 3 }
CALL F X [1;2]
{ X = 3 }
\end{lstlisting}

\subsection{Separation Logic}

\begin{defn}
The notation $P * R$ (separating conjunction) means that $P$ and $R$ holds for disjoint portions of a heap.
\end{defn}

\subsubsection{Frame rule.}
To reason about the heap we need the frame rule, given in theorem \ref{the:frame-rule}. However, we did not have time to include it in our logic.

\begin{theorem}\label{the:frame-rule}
If the command $c$ does not modify any free variable in $R$ then it holds that.
\begin{equation}
\frac
{\{P\}\;c\;\{Q\}}
{\{P * R\}\; c \; \{Q * R\}}
\end{equation}
\end{theorem}

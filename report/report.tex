\documentclass[citeauthoryear]{llncs} %

\usepackage{graphicx}
\usepackage{url}
\usepackage[T1]{fontenc}
\usepackage[utf8]{inputenc}
\usepackage{listings}
\let\proof\relax
\usepackage{amsmath,xfrac,amssymb}
\let\endproof\relax
\usepackage{amsthm}
\usepackage{nameref}

\theoremstyle{definition}
\newtheorem*{notation}{Notation}
\newtheorem{defn}{Definition}
\newtheorem{prop}{Proposition} 

\theoremstyle{notation}

% The type command.
\newcommand{\type}[1] {
	\texttt{#1}
}

%\bibliographystyle{splncs}

\title{SASP F2014}

\titlerunning{}

\author{Simon Bang Terkildsen\\Zhaowei Ding\\Jacob Fischer}

\institute{IT University of Copenhagen}

\begin{document}

\maketitle

\begin{abstract}
We describe how the language IMP is extended with exceptions, function calls and a heap. Furthermore, we define a logic that can reason about the extended IMP language by using Hoare logic as a starting point. The concept of Hoare triples enables us to reason about programs written in the extended IMP, formally verifying a concrete example that demonstrates the use of exceptions and function calls.
\end{abstract}

%!TEX root = ../report.tex

\section{Introduction}
The purpose of this project has been to extend the language IMP, as implemented during the exercise chapters found in Software Foundations[\cite{soft}], with three new features, we will refer to the new language as IMP+. Furthermore, the purpose of the project was to provide the necessary formal constructs for verifying programs written in IMP+. Thus, IMP+ is a superset of IMP, but for the fact that we have excluded the (while-) break command.

The three features we have implemented in IMP+ are:
\begin{itemize}
\item A memory heap, realized through the introduction of a separation logic to the current implementation of IMP
\item Functions and calls, self-contained pieces of code with a local state that can be called any block of code.
\item Exception handling, which allows try-catch statements to "save" the program from an exceptional state that has occurred as the result of the throw command
\end{itemize}

We have built upon the IMP implementation expressed in the formal proof management language Coq. We have extended the implementation, and as such our final product takes the form of a number of Coq source files.

The outline of this report is
\begin{itemize}
\item Introduction, this section.
\item Types and Commands; introduces the types and expressions used in our implementation of IMP+ and the logic.
\item Operational Semantics; describes the operational semantics.
\item Logic; introduces our logic.
\item Hoare Rules; introduces the Hoare rules in our logic.
\item Discussion; discusses interesting points and shortcomings in our implementation of IMP+ and logic.
\end{itemize}

Listing \ref{lst:imp-plus-ex} shows a program written in IMP+. This example will be verified in the Discussion section at the end of the report, following the formal definitions of operational semantics and Hoare rules.

\begin{lstlisting}[mathescape=true,keepspaces=true,label=lst:imp-plus-ex,caption=A program written in IMP+ demonstrating assignment and throw/catch.]
(* Body of the function F *)
Y ::= X;;
TRY
  Y ::= 4;;
  THROW T, [21 - Y];;
  THROW U, []
CATCH T, [Z] DO
  X ::= Y + Z
END

(* The environment, containing a single function F witch
   takes a parameter and has the body defined above.
   The return value is the value of X, in the scope of the 
   body after its execution *)
$\rho$ = {F $\mapsto$ (body, [X], X)}

(* The program entry point *)
X ::= 5;;
CALL F X [X]
\end{lstlisting}

\subsection{Basics of IMP}
IMP is a Turing-complete imperative language. The version of it that we are concerned with is the one implemented in Coq as part of the exercise chapters in Software Foundations. It consists of commands (i.e. statements) and two types of expressions, arithmetic and boolean. Arithmetic and boolean expressions are evaluated recursively, and commands are evaluated with an inductive relation.

Formal definitions of arithmetic and boolean expressions can be found in section \ref{sec:types-com} \nameref{sec:types-com}. Definitions of the various commands as well as the operational semantics for the entirety of IMP+ follow in a later chapter, which will shed more light on the details of how each individual command is evaluated.

\subsection{Function calls}

Creating even moderately complex programs requires a way to structure the code. In programming languages, functions serve this purpose. Though functions are often organized through the usage of class or module constructs, such constructs are not part of IMP+.

Extending IMP with function calls allows a program to be structured into small pieces of code with a single purpose. To implement function calls we introduce scope, such that each function can only read and write to its own state. This is similar to the widely used stack frames, but differ in that the state of a function is not a stack but a function space.

\subsection{Exception handling}

The benefit of extending IMP with exception handling lies in allowing the program to raise and respond to exceptional events during execution. These exceptions are intended to be used in response to events that typically require special processing outside the normal flow of the program's execution.

To this end, we introduce two concepts: the throwing of exceptions, as well as its inverse, catching them. We also introduce the concept of an exceptional state. The program enters an exceptional state the moment an exception is thrown, and leaves it only when that exact type of exception is caught again.

When the program is in an exceptional state, all commands are semantically identical to \verb|SKIP|. This means that a control block such as a loop, if-statement or function is immediately exited upon encountering an exception (at least conceptually, see section \ref{sec:opsem} \nameref{sec:opsem} for details). The exception escapes through all  constructs until it is either caught or the program exits in this exceptional state.

When an exception is thrown, it can be associated with a list of arithmetic expressions that will be evaluated in the immediate state of the program. The resulting list of values will be passed on to where the exception is caught, where they can be assigned to a list of variables and used in the handling logic. This is useful for transmitting information about the exceptional event itself, as the handling logic will otherwise execute from the state that the program was in \emph{before} the code block that threw the exception.

\subsection{Heap}
In computer science, separation logic [\cite{wiki-sep}] is an extension of Hoare logic [\cite{wiki-hoare}], a way of reasoning about programs. The separation logic is for local reasoning on the shared, mutable data structures. In order to extend the imperative programming language (Imp) with shared mutable state and heap.

The goal of the separation logic is to facilitate reasoning about shared mutable data structures, i.e., structures where updatable fields can be referenced from more than one point [\cite{wiki-sep}].  And the separation logic gives a local reasoning.

Extending the imp language with heap, we define syntax and operational semantics  for 4 foundational constructs: allocate, read, write and free in the imp language for manipulating the shared mutable state and heap with also hoare rules for these.


\section{Types and Commands}

Arithmetic expressions, boolean expressions and commands in IMP are given by the types \verb|aexp|, \verb|bexp| and \verb|com|, respectively. In addition, we have introduced three new concepts to IMP: exceptions, heap and functions.

Variables, exceptions, and functions are identified by instances of the types \verb|id|, \verb|exid| and \verb|funid|, respectively. Values are given as all natural numbers $v\in\mathbb{N}$. A tilde above a symbol indicates a list.

A state $\sigma$ is a function space $\verb|id| \rightarrow \mathbb{N}$

A heap $\omega$ is a partial function space $\mathbb{N} \rightharpoonup \mathbb{N}$

An exception $\varepsilon$ is defined as the disjoint union $(\verb|exid|\times\tilde{\mathbb{N}})\uplus\bot$

A function is represented by the triple $\verb|func|: \verb|com| \; \times \; \tilde{id} \; \times \; \verb|aexp|$

Below follows the definitions of arithmetic and boolean expressions, as well as all commands, including the ones we have added for IMP+.

\begin{equation}
\verb|aexp| \triangleq \mathbb{N} \;|\; \verb|id| \;|\; \verb|aexp| + \verb|aexp| \;|\;  \verb|aexp| - \verb|aexp| \;|\;  \verb|aexp| * \verb|aexp|
\end{equation}

\begin{equation}
\verb|bexp| \triangleq True \;|\; False \;|\; \verb|aexp| = \verb|aexp| \;|\; \verb|aexp| \leq \verb|aexp| \;|\; \neg\verb|bexp| \;|\; \verb|bexp| \wedge \verb|bexp|
\end{equation}

\begin{equation}
\begin{alignedat}{2}
\verb|com| \triangleq & \textbf{SKIP} \;|\; & (skip) \\&
	\verb|id| ::= \verb|aexp| \;|\; & (assignment) \\&
	\verb|com|\; ;;\; \verb|com| \;|\; & (sequence) \\&
	\textbf{IFB}\; \verb|bexp|\; \textbf{THEN}\; \verb|com|\; \textbf{ELSE}\; \verb|com|\; \textbf{FI} \;|\; & (if-then-else) \\&
	\textbf{WHILE}\; \verb|bexp|\; \textbf{DO}\; \verb|com|\; \textbf{END}\; \;|\; & (while-loop) \\&
	\textbf{THROW}\; \verb|exid|, \tilde{aexp} \;|\; & (throw) \\&
	\textbf{TRY}\; \verb|com|\; \textbf{CATCH}\; \verb|exid|, \tilde{id}\; \textbf{DO}\; \verb|com|\; \textbf{END} \;|\; & (try-catch) \\&
	\textbf{CALL}\; \verb|funid|\; \verb|id|\; \tilde{aexp} \;|\; & (call) \\&
	\verb|id|\leftarrow\#\;\textbf{ALLOC} \;|\; & (allocate) \\&
	\verb|id|\leftarrow*\;[\verb|aexp|] \;|\; & (read) \\&
	[\verb|aexp|]\leftarrow@\;\verb|aexp| \;|\; & (write) \\&
	\textbf{FREE}\;\verb|id| & (free)
\end{alignedat}
\end{equation}


\section{Operational Semantics}\label{sec:opsem}

\begin{notation}
Let $c$ be a command and $\sigma'$, $\varepsilon$ and $\omega'$ be the state, exception and heap after the execution of $c$ in $\rho$, $\sigma$ and $\omega$.

This is written:
\begin{equation}
(c,\rho,\sigma,\omega) \Downarrow (\sigma',\epsilon,\omega')
\end{equation}

Of note is the fact that the environment $\rho$ remains constant before and after execution, so it is only present on the left-hand side of the evaluation.

Additionally, a program cannot begin execution in an exceptional state, but it can end in one. Thus, the exception $\varepsilon$ is only present on the right-hand side of the evaluation. Conceptually speaking, an individual command \emph{can} begin its execution in an exceptional state (and thus be skipped), but semantically this is modelled by the operational semantics of the $sequence$ command, and not directly in the evaluation relation.
\end{notation}

\begin{notation}
Let $a$ be an arithmetic expression and $n\in\mathbb{N}$ be the result of evaluating $a$ in $\sigma$.

This is written:
\begin{equation}
(a,\sigma) \Downarrow n
\end{equation}
\end{notation}

\begin{notation}
Let $b$ be a boolean expression and $\beta$ the boolean result of evaluating $b$ in $\sigma$.

This is written:
\begin{equation}
(b,\sigma) \Downarrow \beta
\end{equation}
\end{notation}

\begin{defn}\label{def:state-update}
Let $\sigma'$ be the state which is the result of updating the state $\sigma$ with $X = n$, where $X$ is an identifier and $n$ a natural number. It holds that

\begin{equation}
\begin{split}
&\sigma' = \sigma[\sfrac{X}{n}] =>\\
&\sigma'(X) = n \wedge (\forall Y.\; Y \neq X => \sigma'(Y) = \sigma(Y))
\end{split}
\end{equation}
\end{defn}

\begin{defn}\label{def:state-list-update}
Let $\sigma'$ be the state which is the result of updating the state $\sigma$ with $X_i = n_i$ for all $(X : \verb|id|) \in \tilde{X}$ and $n \in (\tilde{n} \cup \mathbb{N})$. It holds that

\begin{equation}
\begin{split}
&\sigma' = \sigma[\sfrac{\tilde{X}}{\tilde{n}}] =>\\
&(\forall X_i \in \tilde{X}.\; \sigma'(X_i) = n_i) \wedge (\forall Y.\; Y \notin \tilde{X} => \sigma'(Y) = \sigma(Y))
\end{split}
\end{equation}
\end{defn}

Below follows the operational semantics for all commands in IMP+, with explanatory text where it is relevant.

\begin{equation}
\frac{}{(\textbf{SKIP}, \rho, \sigma, \omega) \Downarrow (\sigma, \bot, \omega)}
\end{equation}

\begin{equation}\label{eqn:op-assign}
\frac{(a, \sigma) \Downarrow n \qquad \sigma' = \sigma[\sfrac{X}{n}]}{((X ::= a), \rho, \sigma, \omega) \Downarrow (\sigma', \bot, \omega)}
\end{equation}

\begin{equation}
\frac{(b, \sigma) \Downarrow true \qquad (c_1, \rho, \sigma, \omega) \Downarrow (\sigma', \varepsilon, \omega')}{((\textbf{IFB}\; b\; \textbf{THEN}\; c_1\; \textbf{ELSE}\; c_2\; \textbf{FI}), \rho, \sigma, \omega) \Downarrow (\sigma', \varepsilon, \omega')}
\end{equation}

\begin{equation}
\frac{(b, \sigma) \Downarrow false \qquad (c_2, \rho, \sigma, \omega) \Downarrow (\sigma', \varepsilon, \omega')}{((\textbf{IFB}\; b\; \textbf{THEN}\; c_1\; \textbf{ELSE}\; c_2\; \textbf{FI}), \rho, \sigma, \omega) \Downarrow (\sigma', \varepsilon, \omega')}
\end{equation}

The operational semantics for $sequence$ are largely the same as in IMP, although one additional case has been added. The original case (\ref{eqn:op-seq}) is the same, except for ending in an arbitrary exception, as with the if-statements. The second case (\ref{eqn:op-seq-exn}), however, is central to the handling of exceptions in IMP+. It is through this that commands which "begin" in an exceptional state, by virtue of the first leg of the sequence ending in one, are skipped.

\begin{equation}\label{eqn:op-seq}
\frac{(c_1, \rho, \sigma, \omega) \Downarrow (\sigma', \bot, \omega') \qquad (c_2, \rho, \sigma', \omega') \Downarrow (\sigma'', \varepsilon, \omega'')}{((c_1\; ;;\; c_2), \rho, \sigma, \omega) \Downarrow (\sigma'', \varepsilon, \omega'')}
\end{equation}

\begin{equation}\label{eqn:op-seq-exn}
\frac{(c_1, \rho, \sigma, \omega) \Downarrow (\sigma', \varepsilon, \omega') \qquad \varepsilon \neq \bot}{((c_1\; ;;\; c_2), \rho, \sigma, \omega) \Downarrow (\sigma', \varepsilon, \omega')}
\end{equation}

The operational semantics for the while-loop have been extended with a case (\ref{eqn:op-while-exn}) for breaking out of the loop if the inner code block results in an exception. The semantics simply state that if the condition is true, and the inner code block results in a state and some exception, the entire statement results in that state and exception. This is the same behavior as for the $break$ command that was implemented in IMP during one exercise session.

\begin{equation}
\frac{(b, \sigma) \Downarrow false}{((\textbf{WHILE}\; b\; \textbf{DO}\; c\; \textbf{END}), \rho, \sigma, \omega) \Downarrow (\sigma, \bot, \omega)}
\end{equation}

\begin{equation}
\frac{(b, \sigma) \Downarrow true \qquad (c, \rho, \sigma, \omega) \Downarrow (\sigma', \bot, \omega') \qquad ((\textbf{WHILE}\; b\; \textbf{DO}\; c\; \textbf{END}), \rho, \sigma', \omega') \Downarrow (\sigma'', \bot, \omega'')}{((\textbf{WHILE}\; b\; \textbf{DO}\; c\; \textbf{END}), \rho, \sigma, \omega) \Downarrow (\sigma'', \bot, \omega'')}
\end{equation}

\begin{equation}\label{eqn:op-while-exn}
\frac{(b, \sigma) \Downarrow true \qquad (c, \rho, \sigma, \omega) \Downarrow (\sigma', \varepsilon, \omega') \qquad \varepsilon \neq \bot}{((\textbf{WHILE}\; b\; \textbf{DO}\; c\; \textbf{END}), \rho, \sigma, \omega) \Downarrow (\sigma', \varepsilon, \omega')}
\end{equation}

\begin{equation}\label{eqn:op-throw}
\frac{(\tilde{a}, \sigma) \Downarrow \tilde{n}}{((\textbf{THROW}\; e,\; \tilde{a}), \rho, \sigma, \omega) \Downarrow (\sigma, (e, \tilde{n}), \omega)}
\end{equation}

The last case (\ref{eqn:op-catch}) for $try-catch$ is the interesting one. The catch clause contains the name of an exception and a list of variable names. If an exception with a matching name is thrown within the first code block, the values associated with the exception are assigned (in linear order) to the variable names in the list and the second block is executed from this updated version of the original state. Otherwise, the second code block is skipped (case \ref{eqn:op-try} and \ref{eqn:op-try-exn}).

\begin{equation}\label{eqn:op-try}
\frac{(c_1, \rho, \sigma, \omega) \Downarrow (\sigma', \bot, \omega')}{((\textbf{TRY}\; c_1\; \textbf{CATCH}\; e,\; \tilde{X}\; \textbf{DO}\; c_2\; \textbf{END}), \rho, \sigma, \omega) \Downarrow (\sigma', \bot, \omega')}
\end{equation}

\begin{equation}\label{eqn:op-try-exn}
\frac{e \neq e' \qquad (c_1, \rho, \sigma, \omega) \Downarrow (\sigma', (e, \tilde{n}), \omega')}{((\textbf{TRY}\; c_1\; \textbf{CATCH}\; e',\; \tilde{X}\; \textbf{DO}\; c_2\; \textbf{END}), \rho, \sigma, \omega) \Downarrow (\sigma', (e, \tilde{n}), \omega')}
\end{equation}

\begin{equation}\label{eqn:op-catch}
\frac{(c_1, \rho, \sigma, \omega) \Downarrow (\sigma', (e, \tilde{n}), \omega') \qquad (c_2, \rho, \sigma[\sfrac{\tilde{X}}{\tilde{n}}], \omega) \Downarrow (\sigma'', \varepsilon, \omega'')}{((\textbf{TRY}\; c_1\; \textbf{CATCH}\; e,\; \tilde{X}\; \textbf{DO}\; c_2\; \textbf{END}), \rho, \sigma, \omega) \Downarrow (\sigma'', \varepsilon, \omega'')}
\end{equation}

A call consists of two identifiers, one for the function to call and one for the variable the return value is stored in, and a list of arithmetic expressions which are the arguments to the function.

As previously stated, a function is defined by a triple containing a code block, the body, a list of identifiers for the parameters, and an arithmetic expression, the return expression, which calculates the return value.

When a function is called, the arguments of the call are evaluated and a new state is constructed by assigning the argument values to the parameters (in linear order) of the function, in the empty state. The body is executed in the constructed state. The return expression is evaluated in the state the execution of the body terminates in. The return value are stored in the variable for the same scope(state) of the call.

Note, the return expression is evaluated and stored in the specified variable regardless of whether an exception was the reason for the termination of the body or not. However, if the body terminates with an exception, then the call terminates in the same exception.

The heap is the same in both the scope of the call and the body of the called function. Any effect on the heap in the body will persist after the call terminates, even if it terminates with an exception. (But the catch clause will discard any changes made to the heap in the corresponding try block)

\begin{equation}
\frac
{(\tilde{a},\sigma) \Downarrow \tilde{v} \quad \sigma''= emp[\sfrac{\tilde{v}}{\tilde{p}}] \quad (c, \rho, \sigma'', \omega) \Downarrow (\sigma''', \varepsilon, \omega') \quad (r,\sigma''') \Downarrow w \quad \sigma' = \sigma[\sfrac{X}{w}] \quad \rho(f) = (c, \tilde{p}, r)}
{((\textbf{CALL}\; F \; X  \; \tilde{a}), \rho, \sigma, \omega) \Downarrow (\sigma', \varepsilon, \omega')}
\end{equation}

\begin{equation}
\frac{\omega(m)=\bot \qquad \omega'=\omega\cup\{m \rightarrow 0\} \qquad \sigma' = \sigma[\sfrac{X}{m}]}{((\textbf{ALLOC}\;X), \rho, \sigma, \omega) \Downarrow (\sigma', \bot, \omega')}
\end{equation}

\begin{equation}
\frac{(a,\sigma) \Downarrow m \qquad \omega(m)=v \qquad \sigma'=\sigma[\sfrac{X}{v}]}{((X\leftarrow*\;a), \rho, \sigma, \omega) \Downarrow (\sigma', \bot, \omega)}
\end{equation}

\begin{equation}
\frac{(a,\sigma)\Downarrow m \qquad (a',\sigma)\Downarrow v \qquad \omega'=(\omega\backslash\{m\rightarrow\omega(m)\})\cup\{m\rightarrow v\}}{((a\leftarrow@\;a'), \rho, \sigma, \omega) \Downarrow (\sigma, \bot, \omega')}
\end{equation}

\begin{equation}
\frac{\sigma(X)=m \qquad \omega(m)\neq\bot \qquad \omega'=\omega\backslash\{m\rightarrow\omega(m)\}}{((\textbf{FREE}\;X), \rho, \sigma, \omega) \Downarrow (\sigma, \bot, \omega')}
\end{equation}

\section{Logic}
\subsection{Hoare Logic}
To reason about a code block we introduce Hoare logic. To this end we define an \verb|assertion| to be a predicate, $P$, which takes a state, an exception, and a heap.

Hoare triples are the core of Hoare logic, it allows us to describe the effects of executing a piece of code. It has the form $\{P\}\;c\;\{Q\}$ where $P,Q : \verb|assertion|$ and $c : \verb|com|$.
Equation \ref{eqn:hoare_triple_def} defines the Hoare triple.

\begin{equation}\label{eqn:hoare_triple_def}
\begin{alignedat}{1}
\{P\}\;c\;\{Q\} \triangleq &\forall \sigma \: \sigma' \: \omega \: \omega' \: \epsilon .\\
& (c,\sigma,\omega) \Downarrow (\sigma', \epsilon, \omega') \Rightarrow\\
& P(\sigma,\bot,\omega) \Rightarrow\\
& Q(\sigma',\epsilon,\omega')
\end{alignedat}
\end{equation}

Listing \ref{lst:hoare_ex_asgn} shows an example of using Hoare logic to verify a simple program in IMP.

\begin{lstlisting}[mathescape=true,keepspaces=true,label=lst:hoare_ex_asgn,caption=A simple code block proven using Hoare triples.]
{ 1 = 1               } $\text{This is trivially true, but necessary due to }$
                        $\text{how the assignment Hoare rule (\ref{eqn:hoare-assign}) work}$
X ::= 1                 $\text{Updates the state with X = 1 per the}$
                        $\text{operational semantics for assignment (\ref{eqn:op-assign})}$
{ X = 1               } $\text{Trivially true; the state has just been updated}$
                        $\text{such that X = 1}$
{ X = 1 $\wedge$ X $\cdot$ 10 = 10 } $\text{The right operand of the conjunction is}$
                        $\text{introduced to conform to the assignment Hoare rule}$
Y ::= X $\cdot$ 10
{ X = 1 $\wedge$      Y = 10 } $\text{True; the state has jst been updated such}$
                        $\text{that Y = 10}$
\end{lstlisting}

\subsection{Specification Logic}
The specification logic allows us to reason about functions and function calls. We introduce this logic by extending the Hoare logic.

The \verb|assertion| is extended such that it takes $\rho$ as an additional parameter. The Hoare triple is extended such that it is a function from $\rho$ to the Hoare triple defined in equation \ref{eqn:hoare_triple_def}.
\begin{equation}
\begin{alignedat}{2}
\{P\}\;c\;\{Q\} \triangleq \forall \sigma \: \sigma' \: \omega \: \omega' \: \epsilon .\\
 \lambda\rho.(
  && (c,\rho,\sigma,\omega) \Downarrow (\sigma', \epsilon, \omega') \Rightarrow\\
&& P(\rho,\sigma,\bot,\omega) \Rightarrow\\
&& Q(\rho,\sigma',\epsilon,\omega'))
\end{alignedat}
\end{equation}

Listing \ref{lst:spec-ex-body} through \ref{lst:spec-ex-prog} is an example of a program which calls a function and assigns the return value to a variable in the local state.

\begin{lstlisting}[mathescape=true,keepspaces=true,label=lst:spec-ex-body,caption=The body of the function F]
{ X + Y = X + Y }
{Z := X + Y}
{ Z = X + Y     }
\end{lstlisting}

\begin{lstlisting}[mathescape=true,keepspaces=true,label=lst:spec-ex-env,caption=A partial function space containing the function F.]
$\rho$ = $\{$F $\rightarrow$ (Z := X + Y, [X,Y], Z)$\}$
\end{lstlisting}

\begin{lstlisting}[mathescape=true,keepspaces=true,label=lst:spec-ex-prog,caption=A program which call the function F and stores the result in X]
{ 3 = 3 }
CALL F X [1;2]
{ X = 3 }
\end{lstlisting}

%!TEX root = ../report.tex

\section{Hoare Rules}

Every Hoare rule in this section has been proven formally in Coq, with the exception of rules \ref{eqn:hoare-alloc}, \ref{eqn:hoare-read}, \ref{eqn:hoare-write} and \ref{eqn:hoare-free}. The proofs can be found ./src/Hoare.v .

\begin{defn}

Let $e$ be address to an allocated cell on the heap and $e'$ the value stored in the cell.

\begin{equation}
e \mapsto e'
\end{equation}

To state that an address points to an allocated cell without caring about the value, we write
\begin{equation}
e \mapsto \textbf{-}
\end{equation}

\end{defn}

\begin{defn}
\verb|emp| is defined to be an assertion which holds for the empty heap.
\end{defn}

\begin{defn}
\verb|empty| is defined to be the empty state. However, a state is a (complete) function space from natural numbers. When referring to an empty state, we actually refer to a state, $\sigma$, such that $\forall n \in \mathbb{N}.\; \sigma(n) = 0$.
\end{defn}

\begin{equation}\label{eqn:hoare-cons-pre}
\frac{\vdash\{P'\} \; c \; \{Q\} \qquad P \Rightarrow P'}{\vdash\{P\} \; c \; \{Q\}}
\end{equation}

\begin{equation}\label{eqn:hoare-cons-post}
\frac{\vdash\{P\} \; c \; \{Q'\} \qquad Q \Rightarrow Q'}{\vdash\{P\} \; c \; \{Q\}}
\end{equation}

\begin{equation}\label{eqn:hoare-assign}
\frac{}{\vdash \{Q[\sfrac{X}{a}]\} \; (X ::= a) \; \{Q \wedge \lambda \varepsilon\:\sigma\:\omega. \varepsilon = \bot\}}
\end{equation}

\begin{equation}\label{eqn:hoare-seq}
\frac{\vdash \{P\} \; c_1 \; \{Q \wedge \lambda \varepsilon\:\sigma\:\omega. \varepsilon = \bot\} \qquad \vdash \{Q\} \; c_2 \; \{R\}}{\vdash \{P\} \; (c_1\; ;;\; c_2) \; \{R\}}
\end{equation}

\begin{equation}\label{eqn:hoare-seq-exn}
\frac{\vdash \{P\} \; c_1 \; \{Q \wedge \lambda \varepsilon\:\sigma\:\omega. \varepsilon \neq \bot\}}{\vdash \{P\} \; (c_1\; ;;\; c_2) \; \{R\}}
\end{equation}

\begin{equation}
\frac{\vdash \{P \wedge b=true\} \; c_1 \; \{Q\} \qquad \vdash \{P \wedge b=false\} \; c_2 \; \{Q\}}{\vdash \{P\} \; (\textbf{IFB}\; b\; \textbf{THEN}\; c_1\; \textbf{ELSE}\; c_2\; \textbf{FI}) \; \{Q\}}
\end{equation}

\begin{equation}
\frac{\vdash \{P \wedge b = true\} \; c \; \{P\}}{\vdash \{P\} \; (\textbf{WHILE}\; b\; \textbf{DO}\; c\; \textbf{END}) \; \{P \wedge \lambda \varepsilon\:\sigma\:\omega.((\varepsilon = \bot) \Rightarrow (b = false))\}}
\end{equation}

\begin{equation}\label{eqn:hoare-throw}
\frac{\tilde{a}, \sigma \Downarrow \tilde{n}}{\vdash \{P\} \; (\textbf{THROW}\; e, \tilde{a}) \; \{P \wedge \lambda \varepsilon\:\sigma\:\omega. \varepsilon = (e, \tilde{n})\}}
\end{equation}

\begin{defn}
Let \verb|disjoint| be a function $(\varepsilon\times\texttt{exid})\rightarrow(\top\cup\bot)$, given by the following equation:
\begin{equation}
\texttt{disjoint}(\varepsilon, e) = \begin{cases}
\top & \mbox{if} \; \varepsilon = \bot \\
e \neq e' & \mbox{if} \; \varepsilon = (e', \_)
\end{cases}
\end{equation}
\end{defn}

\begin{equation}\label{eqn:hoare-try}
\frac{\vdash \{P\} \; c_1 \; \{Q \wedge \lambda \varepsilon\:\sigma\:\omega. \texttt{disjoint}(\varepsilon, e)\}}{\vdash \{P\} \; (\textbf{TRY}\; c_1\; \textbf{CATCH}\; e, \tilde{X}\; \textbf{DO}\; c_2\; \textbf{END}) \; \{Q\}}
\end{equation}

\begin{equation}\label{eqn:hoare-catch}
\frac{\vdash \{P\} \; c_1 \; \{\lambda \varepsilon\:\sigma\:\omega. \varepsilon = (e, \tilde{n})\} \qquad \vdash \{\lambda \varepsilon\:\sigma\:\omega. \exists \sigma'. (P(\varepsilon, \sigma', \omega) \wedge \sigma = \sigma'[\sfrac{\tilde{X}}{\tilde{n}}])\} \; c_2 \; \{Q\}}{\vdash \{P\} \; (\textbf{TRY}\; c_1\; \textbf{CATCH}\; e, \tilde{X}\; \textbf{DO}\; c_2\; \textbf{END}) \; \{Q\}}
\end{equation}

\begin{equation}\label{eqn:hoare-call}
F = (c,\tilde{p},r) \wedge \left\{P\right\} c \left\{Q \wedge w = r\right\} \vdash \left\{\tilde{v} = \tilde{a} \wedge P(\texttt{empty}[\sfrac{\tilde{v}}{\tilde{p}}])\right\}\textbf{CALL} \; F \; X \; \tilde{a} \left\{X = w\right\}
\end{equation}

\begin{equation}\label{eqn:hoare-alloc}
\frac{}{\vdash \{ \texttt{emp} \}\; \textbf{ALLOC}\; X \;\{X \mapsto 0\}}
\end{equation}

\begin{equation}\label{eqn:hoare-read}
\frac{}{\vdash \{\; \lambda\varepsilon\;\sigma\;\omega.(a,\sigma)\Downarrow m \wedge \omega(m) = v\}\; a \rightarrow X \;\{ X = v \}}
\end{equation}

\begin{equation}\label{eqn:hoare-write}
\frac{}{\vdash \{ a_1 \mapsto X \}\; a_1 \leftarrow a_2 \;\{ a_1 \mapsto a_2 \}}
\end{equation}

\begin{equation}\label{eqn:hoare-free}
\frac{}{\vdash \{X \mapsto \textbf{-}\}\textbf{FREE}\;X \{\texttt{emp} \}}
\end{equation}

\section{Discussion}

\subsection{Design Choices}
The operational semantics for SEQUENCE carries the entire burden of propagating exceptions from one command to the next, in case the post-state of one, and thus the implicit pre-state of the next is exceptional. There are a few reasons for doing it this way:

\begin{itemize}
\item It simplifies the evaluation relation, since there is no need to explicitly model the concept of an exceptional pre-state
\item We avoid having to add an additional rule for every command that makes it explicitly behave as SKIP in the presence of an exception
\end{itemize}

This way, any command that can be put in a sequence will implicitly be skipped if the post-state of the first command is exceptional.

\subsection{Verified Examples}

\subsubsection{Caught and Propagated Exception}

\begin{lstlisting}[mathescape=true,keepspaces=true,label=lst:hoare_ex_asgn,caption=A small program demonstrating a try-catch statement.]
{ 3 = 3                         }      $\text{This is trivially true, but necessary due to}$
                                       $\text{how the assignment Hoare rule (\ref{eqn:hoare-assign}) works.}$
X ::= 3;;                              $\text{Updates the state with X = 3 per the}$
                                       $\text{operational semantics for assignment (\ref{eqn:op-assign}).}$
{ X = 3 $\wedge\quad\lambda \varepsilon\:\sigma\:\omega.\:\varepsilon=\bot$            }      $\text{Trivially true; the state has just been updated}$
                                       $\text{such that X = 3.}$
TRY
  { X = 3 $\wedge\quad\lambda \varepsilon\:\sigma\:\omega.\:\varepsilon=\bot$          } $\rightarrow$   $\text{Precondition as per the catch Hoare rule (\ref{eqn:hoare-catch}).}$
  { 4 = 4 $\wedge\quad\lambda \varepsilon\:\sigma\:\omega.\:\varepsilon=\bot$          }      $\text{Implies this by rule of consequence, which is}$
                                       $\text{trivially true.}$
X ::= 4;;                              $\text{Updates the state with X = 4 per the}$
                                       $\text{operational semantics for assignment (\ref{eqn:op-assign}).}$
  { X = 4 $\wedge\quad\lambda \varepsilon\:\sigma\:\omega.\:\varepsilon=\bot$          }
  THROW T, [21 - X];;                  $\text{Puts the program in an exceptional state as}$
                                       $\text{per the operational semantics for throw (\ref{eqn:op-throw}).}$
  { $\lambda \varepsilon\:\sigma\:\omega.\:\varepsilon$ = (T, [17])          }      $\text{Follows from the Hoare rule for throw (\ref{eqn:hoare-throw}).}$
  THROW U, []                          $\text{This statement is skipped by way of the}$
                                       $\text{exception sequence Hoare rule (\ref{eqn:hoare-seq-exn}).}$
  { $\lambda \varepsilon\:\sigma\:\omega.\:\varepsilon$ = (T, [17])          }      $\text{I.e. the exception from before is propagated.}$
CATCH T, [Y] DO
  { $\lambda \varepsilon\:\sigma\:\omega.\:\exists \sigma'.(\sigma(X)=3\wedge\sigma=\sigma'[\sfrac{Y}{17}])$ } $\rightarrow$    $\text{Follows from the catch Hoare rule (\ref{eqn:hoare-catch}).}$
                                       $\text{Any state in which it holds that X = 3}$
                                       $\text{will do as a witness for the existential}$
                                       $\text{quantifier.}$
  { X = 3 $\wedge$ Y = 17              } $\rightarrow$    $\text{The rule of consequence leads us to this.}$
  { X + Y = 20 $\wedge$ Y = 17         }       $\text{Per the assignment Hoare rule (\ref{eqn:hoare-assign}).}$
  X ::= X + Y                           $\text{Updates the state with X = 20}$
  { X = 20 $\wedge$ Y = 17 $\wedge\quad\lambda \varepsilon\:\sigma\:\omega.\:\varepsilon=\bot$}
END
{ X = 20 $\wedge$ Y = 17 $\wedge\quad\lambda \varepsilon\:\sigma\:\omega.\:\varepsilon=\bot$  }      $\text{Follows from the catch Hoare rule (\ref{eqn:hoare-catch}).}$
                                        $\text{This is our desired end state.}$
\end{lstlisting}

%!TEX root = report.tex

\begin{thebibliography}{}
    %\bibitem[2010]{dickens_2010}
    %    Dickens, M. (2010, December). LETTER FREQUENCY. LETTER FREQUENCY. \\
    %    Retrieved from \url{http://mdickens.me/typing/letter_frequency.html}
    \bibitem[1969]{}
    	Hoare, C. A. R. (1969). An axiomatic basis for computer programming. Communications of the ACM, 12(10), 576-580.
\end{thebibliography}

%\appendix

\end{document}
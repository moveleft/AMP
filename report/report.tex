\documentclass[citeauthoryear]{llncs} %

\usepackage{graphicx}
\usepackage{url}
\usepackage[T1]{fontenc}
\usepackage[utf8]{inputenc}
\usepackage{listings}
\let\proof\relax
\usepackage{amsmath,xfrac,amssymb}
\let\endproof\relax
\usepackage{amsthm}
\usepackage{nameref}

\theoremstyle{definition}
\newtheorem*{notation}{Notation}
\newtheorem{defn}{Definition}
\newtheorem{prop}{Proposition} 

\theoremstyle{notation}

% The type command.
\newcommand{\type}[1] {
	\texttt{#1}
}

\bibliographystyle{splncs}

\title{SASP F2014}

\titlerunning{}

\author{Simon Bang Terkildsen\\Zhaowei Ding\\Jacob Fischer}

\institute{IT University of Copenhagen}

\begin{document}

\maketitle

\begin{abstract}
We describe how the language IMP are extended with exceptions and function calls. And, define a logic that can reason about the extended IMP language. By using Hoare logic as a starting point we define our logic to be able to reason about exceptions and function calls.
\end{abstract}

\section{Introduction}
The purpose of this project has been to extend the language IMP, as implemented during the exercise chapters found in Software Foundations, with three new features, a language we will refer to as IMP+ for the remainder of this report. Furthermore, to provide the necessary formal constructs for validating programs written in IMP+.

The three features we have implemented in IMP+ are:
\begin{itemize}
\item A memory heap, realized through the introduction of a separation logic to the current implementation of IMP
\item Function calls, i.e. self-contained pieces of code with a local state that can be run with a dedicated command from inside itself or other pieces of code
\item Exception handling, which allows try-catch statements to “save” the program from an exceptional state that has occurred as the result of the throw command
\end{itemize}

The version of IMP we have built upon is implemented inside the formal proof management language Coq. We have extended this implementation, and as such our final product takes the form of a number of Coq source files.

%!TEX root = ../report.tex

\section{Types and Commands}

\begin{notation}
A tilde above a symbol indicates a list. For example, $\tilde{n}$ could be a list of natural numbers.
\end{notation}

Arithmetic expressions, boolean expressions and commands in IMP are given by the types \verb|aexp|, \verb|bexp| and \verb|com|, respectively. In addition, we have introduced three new concepts to IMP: exceptions, heap and functions.

Variables, exceptions, and functions are identified by instances of the types \verb|id|, \verb|exid| and \verb|funid|, respectively. Values are given as all natural numbers $v\in\mathbb{N}$.

A state $\sigma$ is a function space $\verb|id| \rightarrow \mathbb{N}$

A heap $\omega$ is a partial function space $\mathbb{N} \rightharpoonup \mathbb{N}$

An exception $\varepsilon$ is defined as the disjoint union $(\verb|exid|\times\tilde{\mathbb{N}})\uplus\bot$

A function is represented by the triple $\verb|func|: \verb|com| \; \times \; \tilde{id} \; \times \; \verb|aexp|$

Let $\rho$ be a partial function space, 
$\verb|funid| \rightharpoonup \verb|func|$, 
such that 
$\rho = \{ id_i \rightarrow f_i | id_i : \verb|funid| \wedge f_i:\verb|func| \}$.

Below follows the definitions of arithmetic and boolean expressions, as well as all commands, including the ones we have added for IMP+\footnote{The notation for $allocate$, $read$ and $write$ differs slightly from the syntax in the code.}.

\begin{equation}
\verb|aexp| \triangleq \mathbb{N} \;|\; \verb|id| \;|\; \verb|aexp| + \verb|aexp| \;|\;  \verb|aexp| - \verb|aexp| \;|\;  \verb|aexp| * \verb|aexp|
\end{equation}

\begin{equation}
\verb|bexp| \triangleq True \;|\; False \;|\; \verb|aexp| = \verb|aexp| \;|\; \verb|aexp| \leq \verb|aexp| \;|\; \neg\verb|bexp| \;|\; \verb|bexp| \wedge \verb|bexp|
\end{equation}

\begin{equation}
\begin{alignedat}{2}
\verb|com| \triangleq & \textbf{SKIP} \;|\; & (skip) \\&
	\verb|id| ::= \verb|aexp| \;|\; & (assignment) \\&
	\verb|com|\; ;;\; \verb|com| \;|\; & (sequence) \\&
	\textbf{IFB}\; \verb|bexp|\; \textbf{THEN}\; \verb|com|\; \textbf{ELSE}\; \verb|com|\; \textbf{FI} \;|\; & (if-then-else) \\&
	\textbf{WHILE}\; \verb|bexp|\; \textbf{DO}\; \verb|com|\; \textbf{END}\; \;|\; & (while-loop) \\&
	\textbf{THROW}\; \verb|exid|, \tilde{aexp} \;|\; & (throw) \\&
	\textbf{TRY}\; \verb|com|\; \textbf{CATCH}\; \verb|exid|, \tilde{id}\; \textbf{DO}\; \verb|com|\; \textbf{END} \;|\; & (try-catch) \\&
	\textbf{CALL}\; \verb|funid|\; \verb|id|\; \tilde{aexp} \;|\; & (call) \\&
	\textbf{ALLOC}\;\verb|id| \;|\; & (allocate) \\&
	\verb|aexp| \rightarrow \verb|id|\;|\; & (read) \\&
	\verb|aexp|\leftarrow\;\verb|aexp| \;|\; & (write) \\&
	\textbf{FREE}\;\verb|id| & (free)
\end{alignedat}
\end{equation}

{\scriptsize
Note, to increase clarity, some of the notation for allocation, read and write differs from the syntax in the Coq code. The syntax used in Coq is \verb|id <-# ALLOC|, \verb|id <-* [id]|, and \verb|[aexp] <-@ aexp|
}

\section{Operational Semantics}\label{sec:opsem}

\begin{notation}
Let $c$ be a command and $\sigma'$, $\varepsilon$ and $\omega'$ be the state, exception and heap after the execution of $c$ in $\rho$, $\sigma$ and $\omega$.

This is written:
\begin{equation}
(c,\rho,\sigma,\omega) \Downarrow (\sigma',\epsilon,\omega')
\end{equation}

Of note is the fact that the environment $\rho$ remains constant before and after execution, so it is only present on the left-hand side of the evaluation.

Additionally, a program cannot begin execution in an exceptional state, but it can end in one. Thus, the exception $\varepsilon$ is only present on the right-hand side of the evaluation. Conceptually speaking, an individual command \emph{can} begin its execution in an exceptional state (and thus be skipped), but semantically this is modelled by the operational semantics of the $sequence$ command, and not directly in the evaluation relation.
\end{notation}

\begin{notation}
Let $a$ be an arithmetic expression and $n\in\mathbb{N}$ be the result of evaluating $a$ in $\sigma$.

This is written:
\begin{equation}
(a,\sigma) \Downarrow n
\end{equation}
\end{notation}

\begin{notation}
Let $b$ be a boolean expression and $\beta$ the boolean result of evaluating $b$ in $\sigma$.

This is written:
\begin{equation}
(b,\sigma) \Downarrow \beta
\end{equation}
\end{notation}

\begin{defn}\label{def:state-update}
Let $\sigma'$ be the state which is the result of updating the state $\sigma$ with $X = n$, where $X$ is an identifier and $n$ a natural number. It holds that

\begin{equation}
\begin{split}
&\sigma' = \sigma[\sfrac{X}{n}] =>\\
&\sigma'(X) = n \wedge (\forall Y.\; Y \neq X => \sigma'(Y) = \sigma(Y))
\end{split}
\end{equation}
\end{defn}

\begin{defn}\label{def:state-list-update}
Let $\sigma'$ be the state which is the result of updating the state $\sigma$ with $X_i = n_i$ for all $(X : \verb|id|) \in \tilde{X}$ and $n \in (\tilde{n} \cup \mathbb{N})$. It holds that

\begin{equation}
\begin{split}
&\sigma' = \sigma[\sfrac{\tilde{X}}{\tilde{n}}] =>\\
&(\forall X_i \in \tilde{X}.\; \sigma'(X_i) = n_i) \wedge (\forall Y.\; Y \notin \tilde{X} => \sigma'(Y) = \sigma(Y))
\end{split}
\end{equation}
\end{defn}

Below follows the operational semantics for all commands in IMP+, with explanatory text where it is relevant.

\begin{equation}
\frac{}{(\textbf{SKIP}, \rho, \sigma, \omega) \Downarrow (\sigma, \bot, \omega)}
\end{equation}

\begin{equation}\label{eqn:op-assign}
\frac{(a, \sigma) \Downarrow n \qquad \sigma' = \sigma[\sfrac{X}{n}]}{((X ::= a), \rho, \sigma, \omega) \Downarrow (\sigma', \bot, \omega)}
\end{equation}

\begin{equation}
\frac{(b, \sigma) \Downarrow true \qquad (c_1, \rho, \sigma, \omega) \Downarrow (\sigma', \varepsilon, \omega')}{((\textbf{IFB}\; b\; \textbf{THEN}\; c_1\; \textbf{ELSE}\; c_2\; \textbf{FI}), \rho, \sigma, \omega) \Downarrow (\sigma', \varepsilon, \omega')}
\end{equation}

\begin{equation}
\frac{(b, \sigma) \Downarrow false \qquad (c_2, \rho, \sigma, \omega) \Downarrow (\sigma', \varepsilon, \omega')}{((\textbf{IFB}\; b\; \textbf{THEN}\; c_1\; \textbf{ELSE}\; c_2\; \textbf{FI}), \rho, \sigma, \omega) \Downarrow (\sigma', \varepsilon, \omega')}
\end{equation}

The operational semantics for $sequence$ are largely the same as in IMP, although one additional case has been added. The original case (\ref{eqn:op-seq}) is the same, except for ending in an arbitrary exception, as with the if-statements. The second case (\ref{eqn:op-seq-exn}), however, is central to the handling of exceptions in IMP+. It is through this that commands which "begin" in an exceptional state, by virtue of the first leg of the sequence ending in one, are skipped.

\begin{equation}\label{eqn:op-seq}
\frac{(c_1, \rho, \sigma, \omega) \Downarrow (\sigma', \bot, \omega') \qquad (c_2, \rho, \sigma', \omega') \Downarrow (\sigma'', \varepsilon, \omega'')}{((c_1\; ;;\; c_2), \rho, \sigma, \omega) \Downarrow (\sigma'', \varepsilon, \omega'')}
\end{equation}

\begin{equation}\label{eqn:op-seq-exn}
\frac{(c_1, \rho, \sigma, \omega) \Downarrow (\sigma', \varepsilon, \omega') \qquad \varepsilon \neq \bot}{((c_1\; ;;\; c_2), \rho, \sigma, \omega) \Downarrow (\sigma', \varepsilon, \omega')}
\end{equation}

The operational semantics for the while-loop have been extended with a case (\ref{eqn:op-while-exn}) for breaking out of the loop if the inner code block results in an exception. The semantics simply state that if the condition is true, and the inner code block results in a state and some exception, the entire statement results in that state and exception. This is the same behavior as for the $break$ command that was implemented in IMP during one exercise session.

\begin{equation}
\frac{(b, \sigma) \Downarrow false}{((\textbf{WHILE}\; b\; \textbf{DO}\; c\; \textbf{END}), \rho, \sigma, \omega) \Downarrow (\sigma, \bot, \omega)}
\end{equation}

\begin{equation}
\frac{(b, \sigma) \Downarrow true \qquad (c, \rho, \sigma, \omega) \Downarrow (\sigma', \bot, \omega') \qquad ((\textbf{WHILE}\; b\; \textbf{DO}\; c\; \textbf{END}), \rho, \sigma', \omega') \Downarrow (\sigma'', \bot, \omega'')}{((\textbf{WHILE}\; b\; \textbf{DO}\; c\; \textbf{END}), \rho, \sigma, \omega) \Downarrow (\sigma'', \bot, \omega'')}
\end{equation}

\begin{equation}\label{eqn:op-while-exn}
\frac{(b, \sigma) \Downarrow true \qquad (c, \rho, \sigma, \omega) \Downarrow (\sigma', \varepsilon, \omega') \qquad \varepsilon \neq \bot}{((\textbf{WHILE}\; b\; \textbf{DO}\; c\; \textbf{END}), \rho, \sigma, \omega) \Downarrow (\sigma', \varepsilon, \omega')}
\end{equation}

\begin{equation}
\frac{(\tilde{a}, \sigma) \Downarrow \tilde{n}}{((\textbf{THROW}\; e,\; \tilde{a}), \rho, \sigma, \omega) \Downarrow (\sigma, (e, \tilde{n}), \omega)}
\end{equation}

The last case (\ref{eqn:op-catch}) for $try-catch$ is the interesting one. The catch clause contains the name of an exception and a list of variable names. If an exception with a matching name is thrown within the first code block, the values associated with the exception are assigned (in linear order) to the variable names in the list and the second block is executed from this updated version of the original state. Otherwise, the second code block is skipped (case \ref{eqn:op-try} and \ref{eqn:op-try-exn}).

\begin{equation}\label{eqn:op-try}
\frac{(c_1, \rho, \sigma, \omega) \Downarrow (\sigma', \bot, \omega')}{((\textbf{TRY}\; c_1\; \textbf{CATCH}\; e,\; \tilde{X}\; \textbf{DO}\; c_2\; \textbf{END}), \rho, \sigma, \omega) \Downarrow (\sigma', \bot, \omega')}
\end{equation}

\begin{equation}\label{eqn:op-try-exn}
\frac{e \neq e' \qquad (c_1, \rho, \sigma, \omega) \Downarrow (\sigma', (e, \tilde{n}), \omega')}{((\textbf{TRY}\; c_1\; \textbf{CATCH}\; e',\; \tilde{X}\; \textbf{DO}\; c_2\; \textbf{END}), \rho, \sigma, \omega) \Downarrow (\sigma', (e, \tilde{n}), \omega')}
\end{equation}

\begin{equation}\label{eqn:op-catch}
\frac{(c_1, \rho, \sigma, \omega) \Downarrow (\sigma', (e, \tilde{n}), \omega') \qquad (c_2, \rho, \sigma[\sfrac{\tilde{X}}{\tilde{n}}], \omega) \Downarrow (\sigma'', \varepsilon, \omega'')}{((\textbf{TRY}\; c_1\; \textbf{CATCH}\; e,\; \tilde{X}\; \textbf{DO}\; c_2\; \textbf{END}), \rho, \sigma, \omega) \Downarrow (\sigma'', \varepsilon, \omega'')}
\end{equation}

A call consists of two identifiers, one for the function to call and one for the variable the return value is stored in, and a list of arithmetic expressions which are the arguments to the function.

As previously stated, a function is defined by a triple containing a code block, the body, a list of identifiers for the parameters, and an arithmetic expression, the return expression, which calculates the return value.

When a function is called, the arguments of the call are evaluated and a new state is constructed by assigning the argument values to the parameters (in linear order) of the function, in the empty state. The body is executed in the constructed state. The return expression is evaluated in the state the execution of the body terminates in. The return value are stored in the variable for the same scope(state) of the call.

Note, the return expression is evaluated and stored in the specified variable regardless of whether an exception was the reason for the termination of the body or not. However, if the body terminates with an exception, then the call terminates in the same exception.

The heap is the same in both the scope of the call and the body of the called function. Any effect on the heap in the body will persist after the call terminates, even if it terminates with an exception. (But the catch clause will discard any changes made to the heap in the corresponding try block)

\begin{equation}
\frac
{(\tilde{a},\sigma) \Downarrow \tilde{v} \quad \sigma''= emp[\sfrac{\tilde{v}}{\tilde{p}}] \quad (c, \rho, \sigma'', \omega) \Downarrow (\sigma''', \varepsilon, \omega') \quad (r,\sigma''') \Downarrow w \quad \sigma' = \sigma[\sfrac{X}{w}] \quad \rho(f) = (c, \tilde{p}, r)}
{((\textbf{CALL}\; F \; X  \; \tilde{a}), \rho, \sigma, \omega) \Downarrow (\sigma', \varepsilon, \omega')}
\end{equation}

\section{Logic}
To reason about a program we will introduce hoare triples and a specification logic.

An assertion, $\alpha$, is a function $\omega \rightarrow \epsilon \rightarrow \sigma \rightarrow \verb|Prop|$ 

A hoare triple is define as $\{P\}\;c\;\{Q\}$ where $P,\;Q : \alpha$ and $c : \verb|com|$

To be able to reason about function calls we introduce program, $\rho$, a function space $\verb|funid| \rightarrow \verb|com| \;\times\; [\verb|id|] $. And, we create a specification logic, $f \vdash \{P\} \; c \; \{Q\}$.

\subsection{Hoare Rules}

\begin{equation}
\frac{}{\vdash \{Q[\sfrac{X}{a}]\} \; (X ::= a) \; \{\lambda \varepsilon \sigma. Q(\varepsilon, \sigma) \wedge \varepsilon = \bot\}}
\end{equation}

\begin{equation}
\frac{\vdash \{P\} \; c_1 \; \{\lambda \varepsilon \sigma. Q(\varepsilon, \sigma) \wedge \varepsilon = \bot\} \qquad \vdash \{Q\} \; c_2 \; \{R\}}{\vdash \{P\} \; (c_1\; ;;\; c_2) \; \{R\}}
\end{equation}

\begin{equation}
\frac{\vdash \{P\} \; c_1 \; \{\lambda \varepsilon \sigma. Q(\varepsilon, \sigma) \wedge \varepsilon \neq \bot\}}{\vdash \{P\} \; (c_1\; ;;\; c_2) \; \{R\}}
\end{equation}

\begin{equation}
\frac{\vdash \{P \wedge b=true\} \; c_1 \; \{Q\} \qquad \vdash \{P \wedge b=false\} \; c_2 \; \{Q\}}{\vdash \{P\} \; (IFB\; b\; THEN\; c_1\; ELSE\; c_2\; FI) \; \{Q\}}
\end{equation}

\begin{equation}
\frac{\vdash \{P \wedge b = true\} \; c \; \{P\}}{\vdash \{P\} \; (WHILE\; b\; DO\; c\; END) \; \{P \wedge b = false\}}
\end{equation}

\begin{equation}
\frac{\tilde{a}, \sigma \Downarrow \tilde{n}}{\vdash \{P\} \; (THROW\; e, \tilde{a}) \; \{\lambda \varepsilon \sigma. P (\bot, \sigma) \wedge \varepsilon = (e, \tilde{n})\}}
\end{equation}

\begin{equation}
disjoint(\varepsilon, e) = \begin{cases}
\top & \mbox{if} \; \varepsilon = \bot \\
e \neq e' & \mbox{if} \; \varepsilon = (e', \_)
\end{cases}
\end{equation}

\begin{equation}
\frac{\vdash \{P\} \; c_1 \; \{\lambda \varepsilon \sigma. Q(\varepsilon, \sigma) \wedge disjoint(\varepsilon, e)\}}{\vdash \{P\} \; (TRY\; c_1\; CATCH\; e, \tilde{X}\; DO\; c_2\; END) \; \{Q\}}
\end{equation}

\begin{equation}
\frac{\vdash \{P\} \; c_1 \; \{\lambda \varepsilon \sigma. \varepsilon = (e, \tilde{n})\} \qquad \vdash \{\lambda \varepsilon \sigma. \exists \sigma'. (P(\varepsilon, \sigma') \wedge \sigma = \sigma'[\sfrac{\tilde{X}}{\tilde{n}}])\} \; c_2 \; \{Q\}}{\vdash \{P\} \; (TRY\; c_1\; CATCH\; e, \tilde{X}\; DO\; c_2\; END) \; \{Q\}}
\end{equation}

\section{Discussion}

\subsection{Design Choices}

\subsubsection{Exception Handling.}
The operational semantics for $sequence$ carries the entire burden of propagating exceptions from one command to the next, in case the post-state of one, and thus the implicit pre-state of the next is exceptional. There are a few reasons for doing it this way:

\begin{itemize}
\item It simplifies the evaluation relation, since there is no need to explicitly model the concept of an exceptional pre-state
\item We avoid having to add an additional rule for every command that makes it explicitly behave as SKIP in the presence of an exception
\end{itemize}

This way, any command that can be put in a sequence will implicitly be skipped if the post-state of the first command is exceptional.

\subsubsection{Function Calls.}
The operational semantic and rule for a call introduced in this report provide the tools to verify a program, containing functions and calls, written in IMP+. An example of such a program is the one shown in listing \ref{lst:imp-plus-ex}.

However, the semantics and rules are not enough to make it practical to verify a program. The program, in listing \ref{lst:sum-recursive}, can be verified using our logic, but to do so every one of the 20 recursive calls need to be proven separately.

\begin{lstlisting}[mathescape=true,keepspaces=true,label=lst:sum-recursive,caption={A program calculating the sum of 10 and 20, using recursion}.]
(* Body of the function F *)
CALL F X [X + 1; Y - 1]

(* The environment, containing a single function F witch
   takes a parameter and has the body defined above.
   The return value is the value of X, in the scope of the 
   body after its execution *)
$\rho$ = {F $\mapsto$ (body, [X, Y], X)}

(* The program entry point *)
CALL F X [10;20]
\end{lstlisting}

\subsection{Proof of Proposition \ref{prop:program-correct}}

We will now prove the statement given in proposition \ref{prop:program-correct} correct, by decorating the program we saw in listing \ref{lst:imp-plus-ex} with formal assertions. The proof is also given in Coq code in ./src/examples.v (the theorem named $cx\_prog\_correct$).

\begin{lstlisting}[mathescape=true,keepspaces=true,label=lst:hoare_ex_asgn,caption=Decorated variant of the program from listing \ref{lst:imp-plus-ex}.]
$\textbf{Body of F}$
{ X = X                         }      $\text{This is trivially true, but necessary due to}$
                                       $\text{how the assignment Hoare rule (\ref{eqn:hoare-assign}) works.}$
Y ::= X;;                              $\text{Updates the state with Y = X per the}$
                                       $\text{operational semantics for assignment (\ref{eqn:op-assign}).}$
{ Y = X $\wedge\quad\lambda \varepsilon\:\sigma\:\omega.\:\varepsilon=\bot$            }      $\text{This is now our precondition for the try-catch.}$
TRY
  { Y = X $\wedge\quad\lambda \varepsilon\:\sigma\:\omega.\:\varepsilon=\bot$          } $\Rightarrow$   $\text{Precondition as per the catch Hoare rule (\ref{eqn:hoare-catch}).}$
  { 4 = 4 $\wedge\quad\lambda \varepsilon\:\sigma\:\omega.\:\varepsilon=\bot$          }      $\text{Implies this by rule of consequence, which is}$
                                       $\text{trivially true.}$
  Y ::= 4;;                            $\text{Updates the state with X = 4 per the}$
                                       $\text{operational semantics for assignment (\ref{eqn:op-assign}).}$
  { Y = 4 $\wedge\quad\lambda \varepsilon\:\sigma\:\omega.\:\varepsilon=\bot$          }      $\text{Per the assignment Hoare rule (\ref{eqn:hoare-assign}).}$
  THROW T, [21 - Y];;                  $\text{Puts the program in an exceptional state as}$
                                       $\text{per the operational semantics for throw (\ref{eqn:op-throw}).}$
  { $\lambda \varepsilon\:\sigma\:\omega.\:\varepsilon$ = (T, [17])          }      $\text{Follows from the Hoare rule for throw (\ref{eqn:hoare-throw}).}$
  THROW U, []                          $\text{This statement is skipped by way of the}$
                                       $\text{exception sequence Hoare rule (\ref{eqn:hoare-seq-exn}).}$
  { $\lambda \varepsilon\:\sigma\:\omega.\:\varepsilon$ = (T, [17])          }      $\text{I.e. the exception from before is propagated.}$
CATCH T, [Z] DO
  { $\lambda \varepsilon\:\sigma\:\omega.\:\exists \sigma'.(\sigma(Y)=\sigma(X)\wedge\sigma=\sigma'[\sfrac{Z}{17}])$ } $\Rightarrow$ $\text{Follows from the catch Hoare rule (\ref{eqn:hoare-catch}).}$
                                        $\text{The state in which X and Y are both}$
                                        $\text{bound to 5 will do as a witness}$
                                        $\text{for the existential quantifier.}$
  { X = 5 $\wedge$ Y = 5 $\wedge$ Z = 17      } $\Rightarrow$    $\text{The rule of consequence leads us to this.}$
  { Y + Z = 22 $\wedge$ Y = 5 $\wedge$ Z = 17 }       $\text{Rule of consequence again.}$
  X ::= Y + Z                           $\text{Updates the state with X = 22}$
  { X = 22 $\wedge$ Y = 5 $\wedge$ Z = 17 $\wedge\quad\lambda \varepsilon\:\sigma\:\omega.\:\varepsilon=\bot$} $\text{Per the assignment Hoare rule (\ref{eqn:hoare-assign}).}$
END
{ X = 22 $\wedge$ Y = 5 $\wedge$ Z = 17 $\wedge\quad\lambda \varepsilon\:\sigma\:\omega.\:\varepsilon=\bot$} $\Rightarrow$ $\text{Follows from the catch Hoare rule (\ref{eqn:hoare-catch}).}$
{(X = 22 $\wedge$ Y = 5 $\wedge$ Z = 17 $\wedge\quad\lambda \varepsilon\:\sigma\:\omega.\:\varepsilon=\bot$)$\wedge$ 22 = X}
                                         $\text{The above follows from the rule of consequence,}$
                                         $\text{and is required as post-condition of the function}$
                                         $\text{body for the call Hoare rule (\ref{eqn:hoare-call}).}$

$\textbf{Program entry point}$

{ 5 = 5      } $\text{Precondition for assignment Hoare rule (\ref{eqn:hoare-assign}).}$
X ::= 5;;      $\text{The state is updated with X = 5.}$
{ X = 5      } $\text{Precondition for the call Hoare rule (\ref{eqn:hoare-call}).}$
CALL F X [X]   $\text{The value of X is passed as a parameter, and the return expression}$
               $\text{of the function body is bound to X, as by semantics for call (\ref{eqn:op-call}).}$
{ X = 22    }  $\text{Postcondition as per the call Hoare rule (\ref{eqn:hoare-call}).}$
               $\text{Our desired end-state as by proposition \ref{prop:program-correct}.}$
\end{lstlisting}

\subsection{Informal Justification of Admitted Subgoals}
A number of subgoals in the Coq code for the Hoare rules specific to heap operations has been admitted. These are found in the file \verb|./sep-src/Hoare.v|. The goals are named the same as here, where an informal justification for their correctness follows.

\begin{equation}
\begin{split}
\textbf{find\_addr}\qquad\forall a\:b\:v.a\neq b\Rightarrow find b (add a v m) = None \\ \\
\frac{\forall b.a\neq b \qquad m(b)=\bot}{\frac{m'(a)=v\wedge m'(\bar{a})=\bot}{\forall a'.a'\neq a \qquad m'=m[\sfrac{a}{v}]\wedge\neg(\exists a'.m'(a)=v)}}
\end{split}
\end{equation}

When the address $a$ added on the heap with value $v$. The address $a$ is the only one on the heap, we could mapping the address $a$ to the value $v$. Nothing else on the heap.

\begin{equation}
\begin{split}
\textbf{same\_val}\qquad\forall v\:v'.Some\;v = Some\;v' \Rightarrow v=v' \\ \\
\frac{v=v'}{\forall a.m(a)=v\wedge m(a)=v'}
\end{split}
\end{equation}

When the values on the heap with same address, the two values should be the same.

\begin{equation}
\begin{split}
\textbf{update\_stack}\qquad\forall \sigma\:v\:v'.update\;\sigma\;x\;v=\sigma'\Rightarrow\sigma'\;x=v \\ \\
\frac{\sigma'(x)=v}{\sigma'=\sigma[\sfrac{x}{v}]}
\end{split}
\end{equation}

When the variable $x$ update with value $v$ in a new state, we would say that in the new state, the variable $x$ mapping to value $v$.

%!TEX root = report.tex

\begin{thebibliography}{}
    %\bibitem[2010]{dickens_2010}
    %    Dickens, M. (2010, December). LETTER FREQUENCY. LETTER FREQUENCY. \\
    %    Retrieved from \url{http://mdickens.me/typing/letter_frequency.html}
    \bibitem[1969]{}
    	Hoare, C. A. R. (1969). An axiomatic basis for computer programming. Communications of the ACM, 12(10), 576-580.
\end{thebibliography}

%\appendix

\end{document}